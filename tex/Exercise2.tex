\chapter*{Exercise 3}
\section*{Exercise 1: Grad-CAM with TorchCAM (45 mins)}
\textbf{Tools:} torchcam, torchvision \\
Use the package documentation for reference: \\
\url{https://frgfm.github.io/torch-cam/}

\begin{enumerate}
\item Setup environment and load model (15 mins)
\begin{minted}{python}
import torch
from torchcam.methods import GradCAM
from torchvision.models import resnet18
from torchvision.datasets import CIFAR10

# Load pretrained model
model = resnet18(pretrained=True)
model.fc = torch.nn.Linear(512, 10)  # Adapt for CIFAR-10
model.load_state_dict(torch.load('cifar10_resnet18.pth'))
\end{minted}

\item Generate and visualize explanations (30 mins)
\item Play around with the layers you are using for the explanations.
\item What are the differences?
\begin{minted}{python}
from torchcam.utils import overlay_mask
from PIL import Image

# Initialize Grad-CAM
cam_extractor = GradCAM(model, 'layer4')

# Process sample image
img = Image.open("sample_airplane.jpg")
inputs = transform(img).unsqueeze(0)

# Generate heatmap
out = model(inputs)
activation_map = cam_extractor(out.squeeze(0).argmax().item(), out)

# Overlay on image
result = overlay_mask(img, activation_map[0], alpha=0.5)
result.show()
\end{minted}
\end{enumerate}

\section*{Pre-configured Setup}
\begin{minted}{bash}
# Recommended environment
conda create -n xai python=3.8
conda install pytorch torchvision -c pytorch
pip install torchcam captum quantus pillow
\end{minted}

\section*{Sample Solutions Checklist}
\begin{itemize}
\item [ ] Working Grad-CAM visualizations for 3 classes
\item [ ] Faithfulness scores for both methods
\item [ ] Comparative analysis table
\end{itemize}
